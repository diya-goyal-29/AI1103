\documentclass[journal,12pt,twocolumn]{IEEEtran}

\usepackage{setspace}
\usepackage{gensymb}
\singlespacing
\usepackage[cmex10]{amsmath}

\usepackage{amsthm}

\usepackage{mathrsfs}
\usepackage{txfonts}
\usepackage{stfloats}
\usepackage{bm}
\usepackage{cite}
\usepackage{cases}
\usepackage{subfig}

\usepackage{longtable}
\usepackage{multirow}

\usepackage{enumitem}
\usepackage{mathtools}
\usepackage{steinmetz}
\usepackage{tikz}
\usepackage{circuitikz}
\usepackage{verbatim}
\usepackage{tfrupee}
\usepackage[breaklinks=true]{hyperref}
\usepackage{graphicx}
\usepackage{tkz-euclide}

\usetikzlibrary{calc,math}
\usepackage{listings}
    \usepackage{color}                                            %%
    \usepackage{array}                                            %%
    \usepackage{longtable}                                        %%
    \usepackage{calc}                                             %%
    \usepackage{multirow}                                         %%
    \usepackage{hhline}                                           %%
    \usepackage{ifthen}                                           %%
    \usepackage{lscape}     
\usepackage{multicol}
\usepackage{chngcntr}

\DeclareMathOperator*{\Res}{Res}

\renewcommand\thesection{\arabic{section}}
\renewcommand\thesubsection{\thesection.\arabic{subsection}}
\renewcommand\thesubsubsection{\thesubsection.\arabic{subsubsection}}

\renewcommand\thesectiondis{\arabic{section}}
\renewcommand\thesubsectiondis{\thesectiondis.\arabic{subsection}}
\renewcommand\thesubsubsectiondis{\thesubsectiondis.\arabic{subsubsection}}


\hyphenation{op-tical net-works semi-conduc-tor}
\def\inputGnumericTable{}                                 %%

\lstset{
%language=C,
frame=single, 
breaklines=true,
columns=fullflexible
}
\begin{document}

\newcommand{\BEQA}{\begin{eqnarray}}
\newcommand{\EEQA}{\end{eqnarray}}
\newcommand{\define}{\stackrel{\triangle}{=}}
\bibliographystyle{IEEEtran}
\raggedbottom
\setlength{\parindent}{0pt}
\providecommand{\mbf}{\mathbf}
\providecommand{\pr}[1]{\ensuremath{\Pr\left(#1\right)}}
\providecommand{\qfunc}[1]{\ensuremath{Q\left(#1\right)}}
\providecommand{\sbrak}[1]{\ensuremath{{}\left[#1\right]}}
\providecommand{\lsbrak}[1]{\ensuremath{{}\left[#1\right.}}
\providecommand{\rsbrak}[1]{\ensuremath{{}\left.#1\right]}}
\providecommand{\brak}[1]{\ensuremath{\left(#1\right)}}
\providecommand{\lbrak}[1]{\ensuremath{\left(#1\right.}}
\providecommand{\rbrak}[1]{\ensuremath{\left.#1\right)}}
\providecommand{\cbrak}[1]{\ensuremath{\left\{#1\right\}}}
\providecommand{\lcbrak}[1]{\ensuremath{\left\{#1\right.}}
\providecommand{\rcbrak}[1]{\ensuremath{\left.#1\right\}}}
\theoremstyle{remark}
\newtheorem{rem}{Remark}
\newcommand{\sgn}{\mathop{\mathrm{sgn}}}
\providecommand{\abs}[1]{\vert#1\vert}
\providecommand{\res}[1]{\Res\displaylimits_{#1}} 
\providecommand{\norm}[1]{\lVert#1\rVert}
%\providecommand{\norm}[1]{\lVert#1\rVert}
\providecommand{\mtx}[1]{\mathbf{#1}}
\providecommand{\mean}[1]{E[ #1 ]}
\providecommand{\fourier}{\overset{\mathcal{F}}{ \rightleftharpoons}}
%\providecommand{\hilbert}{\overset{\mathcal{H}}{ \rightleftharpoons}}
\providecommand{\system}{\overset{\mathcal{H}}{ \longleftrightarrow}}
	%\newcommand{\solution}[2]{\textbf{Solution:}{#1}}
\newcommand{\solution}{\noindent \textbf{Solution: }}
\newcommand{\cosec}{\,\text{cosec}\,}
\providecommand{\dec}[2]{\ensuremath{\overset{#1}{\underset{#2}{\gtrless}}}}
\newcommand{\myvec}[1]{\ensuremath{\begin{pmatrix}#1\end{pmatrix}}}
\newcommand{\mydet}[1]{\ensuremath{\begin{vmatrix}#1\end{vmatrix}}}
\numberwithin{equation}{subsection}
\makeatletter
\@addtoreset{figure}{problem}
\makeatother
\let\StandardTheFigure\thefigure
\let\vec\mathbf
\renewcommand{\thefigure}{\theproblem}
\def\putbox#1#2#3{\makebox[0in][l]{\makebox[#1][l]{}\raisebox{\baselineskip}[0in][0in]{\raisebox{#2}[0in][0in]{#3}}}}
     \def\rightbox#1{\makebox[0in][r]{#1}}
     \def\centbox#1{\makebox[0in]{#1}}
     \def\topbox#1{\raisebox{-\baselineskip}[0in][0in]{#1}}
     \def\midbox#1{\raisebox{-0.5\baselineskip}[0in][0in]{#1}}
\vspace{3cm}
\title{Assignment 3}
\author{Diya Goyal\\ Roll no. CS20BTECH11014}
\maketitle
\newpage
\bigskip
\renewcommand{\thefigure}{\theenumi}
\renewcommand{\thetable}{\theenumi}
Download latex-tikz codes from 
\begin{lstlisting}
https://github.com/diya-goyal-29/AI1103/blob/main/Assignment%203/Assignment%203.tex
\end{lstlisting}
\section{CSIR-UGC-NET-Exam-JUNE-2015 Question 60 :}\\
Let $X_1, X_2, \dots, X_n$ be independent and identically distributed random variables having an exponential distribution with mean $\frac{1}{\lambda}$.Let $S_n = X_1 + X_2 + \cdots + X_n$ and $N = inf \{n \geq 1: S_n > 1\}$. Then $Var(N)$ is\\
\begin{enumerate}
    \item 1
    \item $\lambda$
    \item $\lambda^2$
    \item $\infty$
\end{enumerate}
\\ \\
\section{Solution :}\\
$Var(N) = \mean{N^2} - \mean{N}^2$\\
Replacing n with n+1\\
inf implies that $N$ must be the least $n$ such that $S_n > 1$.\\
The event $\{N \leq n\}$ implies $\{S_{n+1} > 1\}$ (or equivalently, $\{N > n\}$ implies $\{S_{n+1} \leq 1\}$).\\
\textbf{Tail Sum formula : } A random variable $X$ which takes values only in \mathbf{N}, then $\mean{X} = \sum _{k=1} ^\infty Pr(X\geqk)$\\
\textbf{Proof : }\\
\begin{align}
    \mean{X} &= \sum _{x = 1} ^\infty x Pr(X=x)\\
    &= \sum _{x = 1} ^\infty \sum _{k=1} ^x Pr(X=x)\\
    &= \sum _{k = 1} ^\infty \sum _{x = k} ^\infty Pr(X=x)\\
    &= \sum _{k = 1} ^\infty Pr(X\geq k)
\end{align}

By the tail-sum formula, we can thus get the expectation of $N$ as\\
\begin{align}
    \mean{N} = \sum ^\infty _{n = 0} Pr(N > n) = \sum ^\infty _{n = 0} Pr(S_n \leq 1)
\end{align}
$S_n$ follows Erlang distribution.\\
\begin{align}
    \text{P.D.F of }S_n = f(x)_{S_n} = \frac{\lambda^{n+1} x^{n}e^{-\lambda x}}{(n)!}\\
    Pr(S_n \leq 1) = \int _0 ^1 \frac{\lambda^{n+1} x^{n}e^{-\lambda x}}{(n)!} dx
\end{align}
\textbf{P.D.F of Erlang distribution : }\\
\begin{align}
    f(x) = \frac{\lambda^{k} x^{k-1}e^{-\lambda x}}{(k-1)!}
\end{align}
where,\\
$k$ is the shape\\
$\lambda$ is the rate\\
\textbf{Proof : }We can use mathematical induction to prove this\\
\begin{align}
  S_1 &= X_1\\
  f_{S_1}(x) &= \lambda e^{[-\lambda x]}
\end{align}
Basic step proved.\\
Assuming $S_k$ to be true we need to prove that $S_{k+1}$ is true.\\
\begin{align}
    S_{k+1} &= X_1 + X_2 + X_3 + \dots + X_{k+1}\\
    &= S_k + X_{k+1}\\
    f_{S_{k+1}}(x) &= f_{S_k + X_{k+1}}\\
    &= \int _0 ^x \lambda e^{-\lambda(x-t)} \cdot \lambda^k e^{-\lambda t} \frac{t^{k-1}}{(k-1)!}\\
    &= \lambda^{k+1} e^{-\lambda x} \int _0 ^x \frac{t^{k-1}}{(k-1)!}\\
    &= \lambda^{k+1} e^{-lambda x} \frac{x^k}{(k)!}
\end{align}
Substituting the value\\
\begin{align}
    \mean{N} &= \sum _{n = 0} ^\infty \int _0 ^1  \frac{\lambda^{n+1} x^{n}e^{-\lambda x}}{(n)!} dx\\
    &= \int _0 ^1 \sum _{n = 0} ^\infty \frac{\lambda^{n+1} x^{n}e^{-\lambda x}}{(n)!} dx\\
    e^{x\lambda} &= \sum _{n = 0} ^ \infty\frac{(x\lambda)^n}{(n)!}\\
    \mean{N} &=\lambda  \int _0 ^1 e^{\lambda x} e^{-\lambda x} dx\\
    &= \lambda \int _0 ^1 dx\\
    &= \lambda
\end{align}
By tail-sum formula the expectation of $N^2$ is\\
\begin{align}
    \mean{N^2} &= \sum _{n = 0} ^\infty Pr(N^2 \geq n)\\
    &= \sum _{n = 0} ^\infty [(n+1)^2 - n^2] Pr(N \geq n)\\
    &= \sum _{n = 0} ^\infty [2n + 1] Pr(S_{n+1} \leq 1)\\
    &=  \sum _{n = 0} ^\infty \int _0 ^1 [2n+1] \frac{\lambda^{n+1} x^{n}e^{-\lambda x}}{(n)!} dx\\
    &= \int _0 ^1 \sum _{n = 0} ^\infty [2n+1]\frac{\lambda^{n+1} x^{n}e^{-\lambda x}}{(n)!} dx\\
    &= \int _0 ^1 2\lambda^2 x e^{\lambda x} e^{-\lambda x} + \lambda e^{\lambda x} e^{-\lambda x} dx\\
    &= \lambda ^2 + \lambda
\end{align}
\begin{align}
    Var(N) &= \mean{N^2} - \mean{N}^2\\
    &= \lambda^2 + \lambda - (\lambda)^2\\
    &= \lambda
\end{align}
Thus Option 2 is correct.

\end{document}
