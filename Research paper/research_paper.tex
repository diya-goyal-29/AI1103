\documentclass{beamer}
\usepackage{listings}
\lstset{
%language=C,
frame=single, 
breaklines=true,
columns=fullflexible
}
\usepackage{blkarray}
\usepackage{subcaption}
\usepackage{url}
\usepackage{xurl}
\usepackage{tikz}
\usepackage{tkz-euclide} % loads  TikZ and tkz-base
%\usetkzobj{all}
\usetikzlibrary{calc,math}
\usepackage{float}
\newcommand\norm[1]{\left\lVert#1\right\rVert}
\renewcommand{\vec}[1]{\mathbf{#1}}
\usepackage[export]{adjustbox}
\usepackage[utf8]{inputenc}
\usepackage{amsmath}
\usepackage{amssymb}
\usepackage{wrapfig}
\usepackage{bm}
\usepackage{tikz}
\setbeamertemplate{caption}[numbered]
\usetikzlibrary{automata, positioning}
\usetheme{Boadilla}
\providecommand{\pr}[1]{\ensuremath{\Pr\left(#1\right)}}
\providecommand{\sbrak}[1]{\ensuremath{{}\left[#1\right]}}
\providecommand{\lsbrak}[1]{\ensuremath{{}\left[#1\right.}}
\providecommand{\rsbrak}[1]{\ensuremath{{}\left.#1\right]}}
\providecommand{\pr}[1]{\ensuremath{\Pr\left(#1\right)}}
\providecommand{\brak}[1]{\ensuremath{\left(#1\right)}}
\providecommand{\lbrak}[1]{\ensuremath{\left(#1\right.}}
\providecommand{\rbrak}[1]{\ensuremath{\left.#1\right)}}
\providecommand{\cbrak}[1]{\ensuremath{\left\{#1\right\}}}
\providecommand{\lcbrak}[1]{\ensuremath{\left\{#1\right.}}
\providecommand{\rcbrak}[1]{\ensuremath{\left.#1\right\}}}

\title{Research Paper Presentation}
\author{Diya Goyal}
\date{CS20BTECH11014}
\begin{document}

\begin{frame}
\titlepage
\end{frame}
\begin{frame}
\begin{block} {Title}
Fast Cluster-Learning with prior probability from Big Dataset
\end{block}
\begin{block} {Authors}
\begin{itemize}
    \item Tengyue Li
    \item Simon Fong
    \item Joao Alexandre Lobo Marques
    \item Raymond K Wong
\end{itemize}
\end{block}
\end{frame}
\begin{frame}{Terms to know}
\begin{itemize}
    \item \textbf{Apriori Algorithm : }Apriori is an algorithm for frequent item set mining and association rule learning over relational databases.
\item \textbf{Association Rule Mining : }It is the data mining process of finding the rules that may govern associations and causal objects between sets of items.
\item \textbf{Clustering : }In machine learning, we often group examples as a first step to understand a subject (data set) in a machine learning system. Grouping unlabeled examples is called clustering.
\end{itemize}
\end{frame}
\begin{frame}{Terms to know}
\begin{itemize}
    \item \textbf{Preprocessing : }Data preprocessing is a process of preparing the raw data and making it suitable for a machine learning model. It is the first and crucial step while creating a machine learning model.
\item \textbf{Prior Probability : }A probability as assessed before making reference to certain relevant observations, especially subjectively or on the assumption that all possible outcomes be given the same probability.
\end{itemize}
\end{frame}

\begin{frame}{Introduction}
    \begin{itemize}
        \item The association rule mining algorithms, such as Apriori 
method for instance and other successors were invented 
in the late 90’s. They were not designed to work with big 
data in mind in those days.
\item Apriori algorithm, being a classical association rule mining 
algorithm is known to have a drawback in scalability. The 
time performance in big O notation would have it scale up 
exponentially to the dimensions or the number of attributes 
of the data.
\item In order to tackle the challenges of association rule 
mining over big data or big dataset where the data are 
usually presented as a two-dimensional data matrix, some 
modifications are needed either at the algorithmic level or at 
the data processing level.
    \end{itemize}
\end{frame}
\begin{frame}{}
    \begin{itemize}
 \item Earlier on, the authors advocated an alternative preprocessing method which aims at improving the quality 
of association rules. 
\item It first segments the full dataset into 
clusters prior to running the association rule mining 
algorithm on. The underlying concept is to mine the 
association rules from subsets of the data which are 
supposedly more concentrated in data that occur frequently 
together. This is because clustering helps pulling data that 
are similar to each other together, reducing the distance 
space between the scattered data throughout the whole 
dataset.
\item By this simple idea, the overall association rules 
quality improves. But the problem is, in advance there are 
no way to know which is most appropriate to be mined by 
association rule mining algorithm. 
\item Although the cluster size 
is smaller than the original dataset size, associate rule 
mining algorithm such as Apriori still takes time. 
    \end{itemize}
\end{frame}
\begin{frame}{Proposed Methodology}
    \begin{itemize}
        \item Traditional association rule mining loads the whole 
dataset for generating association rules. 
\item To get around this time-consuming process, the rule generated 
process is controlled by harvesting only a certain number of rules which usually are the top ones.
\item Alternatively, a subset of dataset is used for the association 
rule mining instead of the whole one. In this case, out of 
several clusters or subsets of the original dataset that was 
partitioned by some clustering algorithm, there should be 
one or few out of all clusters should be used for incomplete 
association rule mining.
\item The qualities of the clusters vary as there is 
no rule-of-thumb exists in either assuring the clusters are 
made all suitable for association rule mining or knowing in 
advance which cluster is most suitable for highest quality 
rules to be mined.
    \end{itemize}
\end{frame}

\begin{frame}
\begin{itemize}
    \item As a solution, the proposed process advocates using a statistical property called Prior Probability to be used. Hence it is no longer necessary to try out every cluster, because a user can base on the Prior Probability (PP) to 
decide which cluster or a short-listed set of clusters to be 
selected for association rule mining. Thus speeding up the process.
\item Due to the speed saving, this association rule 
mining strategy is called Fast Cluster Learning (FCL). 
\item FCL divides up the large dataset into small fragments, not randomly, but according to the similar data which shall be grouped together and PP is calculated for each of these clusters.
\item With the PP value in place for each cluster, the cluster 
candidates can be sorted in descending order. The cluster 
candidate that has the highest PP value is suggested to be 
chosen for immediate association rule mining.
\end{itemize}
\end{frame}

\begin{frame}
    \begin{itemize}
        \item In the methodology, parameter free clustering algorithm 
namely Expectation Maximization (EM) algorithm is 
suggested to be used. It will automatically find the optimal 
number of clusters by its maximization mechanism. 
\item  For brevity, the data items only have a 
single numeric attribute whose values distributed Normally 
over all the $k$ clusters, $C_1, C_2 .. C_j.. C_k$
 where $j\epsilon[1,k]$. In a very primitive state of probabilistic clustering, the following are the parameters of the model that need to be calculated: 
 \begin{block}
 $Model\_Param = Model(pp_j, \mu_j, \sigma_j, \text{where } j\epsilon[1,k]) $
 \end{block}
 \item At the beginning the 
Model\_Param values are set randomly; based on the given 
the current parameter values, the probability of the cluster 
membership for each data item is computed; the 
Model\_Param values are re-evaluated and updated using the 
calculated probabilities which are shown to have yielded 
better cluster memberships. This iterative process repeats 
until the Model\_Param appears to converge where no further 
improvement on the cluster membership can be observed. 
    \end{itemize}
\end{frame}

\begin{frame}{Key Phases}
\textbf{Phase 1 : Expectation step}\\
\begin{itemize}
    \item The probability of each data item $x_i$
 where $i\epsilon [1,n]$ 
from the dataset is computed for finding out the best cluster 
membership for the cluster $c_j$
 where $j\epsilon[1,k]$ using
 \begin{block}
 $\pr{E}\forall_{i,j} = \varepsilon_{i,j} = pp_j \times \pr{x_i|c_j}$\\
 where the cluster $pp_j$ is computed in each $j^{th}$ iteration based on the model parameters values and $\pr{x_i|c_j}$ is calculated from the Normal distribution 
of the $j^{th}$ cluster where $j\epsilon[1,k]$ given that the current model is 
configured with the $Model\_Param\epsilon(x_j, \mu_j, \sigma_j)$
 \end{block}
\end{itemize}
\end{frame}

\begin{frame}{}
    \textbf{Phase 2 : Maximization step}
    \begin{itemize}
        \item Calculates the most suitable 
Model\_Param values in order to maximize the likelihood of 
models that are guessed to match the current positions or 
memberships of the data items and the model parameters are re-evaluated according to
\begin{block}
 $\text{Prior probability: } pp_{i,j} = \sum_i \frac{\varepsilon_{i,j}}{n}$\\
Mean: $\mu_{i,j} = \frac{\sum_i x_i \times \varepsilon_{i,j}}{\sum_i \varepsilon_{i,j}}$\\
Standard deviation: $\sigma_{i,j} = \sqrt{\frac{\sum_i \abs{x_i - \mu_j}^2 \times \varepsilon_{i,j}}{\sum_i \varepsilon_{i,j}}}$
\end{block}
    \end{itemize}
\end{frame}

\begin{frame}{}
    \begin{itemize}
        \item These dual steps repeatedly increase the log-likelihood of 
all the clusters until there is no more significant refinement 
according to :
\begin{block}
    $\log\pr{x} = \log \sum_j (\pr{x|c_j} \times pp_j)$
\end{block}
\item Usually the overall quality of the clusters which is 
represented by log-likelihood will rise sharply at the 
beginning over some initial iterations.
\item In our fast cluster learning methodology, EM(Expectation Maximization) 
is chosen because it is guaranteed to converge to highest 
possible log-likelihood fitness.
    \end{itemize}
\end{frame}

\begin{frame}{Limitation}
\begin{itemize}
    \item The only limitation is the 
time consumption which can be significantly large. It is 
because there exist local maximum and global maximum.
\item  For prevention of falling into 
local maximum, the EM algorithm is programmed to run 
several times for obtaining some chances of reaching the 
global maximum as each time they start with different 
random orientation for the initial clusters and model 
parameters, guessing what their suitable parameter values are.
\item Hence there is a compromise e between using a heavy algorithm that 
takes considerable amount of time to run and achieving the 
best clusters most of the time without specifying the k
parameter value.
\end{itemize}
\end{frame}

\begin{frame}{Conclusions}
    
    \begin{itemize}
        \item In this paper, a new data mining methodology called Fast 
Cluster Learning is proposed.
\item It is designed for enhancing the quality of rules by association rule mining such as Apriori method.
\item In this paper, a simple quality indicator called Prior Probability (PP) 
is proposed to use for quickly identifying a cluster that would 
be useful for subsequent rule mining. 
    \end{itemize}
\end{frame}
\end{document}
